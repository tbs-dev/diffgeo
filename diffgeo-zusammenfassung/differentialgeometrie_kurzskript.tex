\documentclass[
pdftex,
oneside,
%halfparskip,
headsepline,
11pt, 
]{scrreprt}

\usepackage[utf8]{inputenc}
\usepackage[ngerman]{babel}
\usepackage{header-commands}

% 1. Emulation von fncychap mit KOMA-Script-Mitteln:
\newlength{\ChapterRuleWidth}
\newcommand*{\ChRuleWidth}[1]{\setlength{\ChapterRuleWidth}{\dimexpr #1}}%
\newcommand*{\ChNameVar}{\setkomafont{chapterprefix}}%
\newcommand*{\ChTitleVar}{\setkomafont{chapter}}%
\newcommand*{\ChNumVar}{\setkomafont{chapternumber}}%
\newcommand*{\ChapterNameCase}[1]{#1}
\newcommand*{\ChNameUpperCase}{\let\ChapterNameCase\MakeUppercase}
\newcommand*{\ChNameIs}{\renewcommand*\ChapterNameCase[1]{##1}}
\newcommand*{\ChNameLowerCase}{\let\ChapterNameCase\MakeLowercase}
\newcommand*{\ChapterTitleCase}[1]{#1}
\newcommand*{\ChTitleUpperCase}{\let\ChapterTitleCase\MakeUppercase}
\newcommand*{\ChTitleIs}{\renewcommand*\ChapterTitleCase[1]{##1}}
\newcommand*{\ChTitleLowerCase}{\let\ChapterTitleCase\MakeLowercase}

% 2. Einstellungen für den Stil Sonny:
\ChRuleWidth{1pt}
\KOMAoptions{chapterprefix}% Es ist ein Präfix-Stil
\newkomafont{chapternumber}{\fontsize{60}{62}\usefont{OT1}{ptm}{m}{n}\selectfont}
\RedeclareSectionCommand[%
beforeskip=-61pt,% Abstand über der Präfixzeile bzw. der Linie
innerskip=15pt,% Abstand zwischen Präfixzeile und Text
afterskip=40pt,% Abstand unter dem Text
font=\normalfont\rmfamily\Huge,% Schrift des Namens
prefixfont=\fontsize{14}{16}\usefont{OT1}{phv}{m}{n}\selectfont,% Schrift der Präfixzeile
]{chapter}
\usepackage{picture}
\usepackage{xcolor}
\renewcommand*{\chapterformat}{%
	\mbox{%
		\setlength{\fboxsep}{0pt}\colorbox{white}{%
			\strut\ChapterNameCase{\chapappifchapterprefix{\enskip}}}%
		{\usekomafont{chapternumber}{%
				\colorbox{white}{\strut\thechapter\IfUsePrefixLine{}{\enskip}}}}%
	}%
}

\renewcommand*{\chapterlineswithprefixformat}[3]{% Ebene, Nummer, Text
	\IfArgIsEmpty{#2}{}{%
		% Die Prefix-Zeile aus Argument 2 wird nur gesetzt, wenn sie vorhanden
		% ist.
		\begin{picture}(0,0)
		\setlength{\linethickness}{\ChapterRuleWidth}%
		\usekomafont{chapternumber}{%
			\put(.5\ChapterRuleWidth,0){%
				\framebox(\dimexpr\linewidth-\ChapterRuleWidth,.9\ht\strutbox){}}}%
		\end{picture}%
		#2%
	}%
	\ChapterTitleCase{#3}%
}

\usepackage{picture}
\usepackage{xcolor}




\author{Tobias Klas}
\title{Zusammenfassung Differentialgeometrie}

\begin{document}
\maketitle	
\tableofcontents

\newpage	

\chapter{Analysis mit Kurven}
\section{Parametrisierte Kurven}
\begin{de}[Parametrisierte Kurven]
	Seien $a,b\in \R$ mit $a<b$ und $X$ ein topologischer Raum. Eine stetige Funktion $c:[a,b]\to X$ heißt \textit{parametrisierte Kurve}. Der Punkt $c(a)\in X$ heißt \textit{Anfangspunkt} und der Punkt  $c(b)\in X$ \textit{Endpunkt} der parametrisierten Kurve $c$.
	Ist $X=\R^n$ und ist $c$ eine $C^k$-Funktion, so nennen wir $c$ ein \textit{parametrisierte Kurve der Klasse $C^k$} oder einfach parametrisierte $C^k$-Kurve.  Eine parametrisierte $C^k$-Kurve $c$ heißt \textit{geschlossen}, falls \[ c(a)=c(b) \quad \textnormal{und falls } k\geq 1 \textnormal{ gilt für alle } 1 \leq r\leq k:\; D^rc(a)=D^rc(b).\]
\end{de}
\begin{de}[Jordan-Kurve] Eine parametrisierte Kurve $c$ heißt \textit{Jordan-Kurve}, falls $c$ geschlossen ist und $c$ auf $[a,b)$ injektiv ist. 
\end{de}
\begin{bsp}\quad
	\begin{itemize}
		\item \textit{(Doppel-)Helix:}
		Sei $\sigma, \rho\in\R$. Das Bild der parametrisierten $C^\infty$-Kurve \[ c_{\rho,\sigma}:\R\to \R^3,\qquad t\mapsto(\rho \cos t,\rho \sin t,\sigma t)\] ist ein \textit{Kreis} um $\vek{0}$ mit Radius $|\rho|$, falls $\sigma=0$. Ist $\sigma \rho\neq 0$, so ist das Bild eine \textit{Helix}. Die Bilder von $c_{-\sigma,\rho}$ und $c_{\sigma,\rho}$ ergeben eine \textit{Doppelhelix}.
		\item \textit{Doppelhelix:}
		\item \textit{Neilsche Parabel:}
	\end{itemize}
\end{bsp}
\begin{de}[Äquivalente Kurven]
	Es seien $c_1:I_1\to \R^n$ und $c_2:I_2\to\R^n$ parametrisierte $C^k$-Kurven. Wir nennen $c_1$ und $c_2$ \textit{linear äquivalent}, falls es eine affin lineare bijektive Abbildung \[\varphi:I_1\to I_2, \quad t\mapsto at+b\] gibt, so dass \[ c_1=c_2\circ\varphi. \] Die Abbildung $\varphi$ heißt \textit{Parametertransformation}. Ist \[\varphi\in C^k,\quad c_1=c_2\circ\varphi\quad \textnormal{und}\quad\dot{\varphi}(t)\neq0 \textnormal{ für alle }t\in I_1,\] so heißen $c_1$ und $c_2$ \textit{äquivalent}. Gilt sogar für alle $t\in I_1$, dass \[ \dot{\varphi}(t)>0,\] so heißen $c_1$ und $c_2$ \textit{orientierbar äquivalent} und $\varphi$ \textit{zulässige Parametertransformation}.  
\end{de}
\begin{lem}
	Durch die lineare Äquivalenz, die Äquivalenz und die orientierbare Äquivalenz zweier parametrisierter Kurven sind Äquivalenzrelationen definiert.
\end{lem}
\begin{proof}
	
\end{proof}
\begin{de}[Länge]
	Sei $(X,d)$ ein metrischer Raum. Dann ist die \textbf{Länge} einer parametrisierten Kurve $c:[a,b]\to X$ definiert durch
	\[ L(c)=\sup \left\{\sum \limits _{{i=1}}^{n}d(c(t_{i}),
	c(t_{{i-1}}))\,{\Bigg |}\,n\in {\mathbb  {N}},a\leq t_{0}<t_{1}<\ldots <t_{n}\leq b\right\}. \]
	Eine parametrisierte Kurve mit endlicher Länge heißt \textit{rektifizierbar}.
\end{de}
\begin{lem}
	Jede parametrisierte $C^k$-Kurve ist rektifizierbar und ihre Länge ist durch \[ L(c)=\int_a^b\norm{\dot{c}(t)}_2\diff{t}\] gegeben.
\end{lem}
\begin{proof}
	Sei $Z:a=t_0<...<t_m=b$ eine Zerlegung von $[a,b]$, so nennen wir \[\delta(Z):=\sup_{1\leq i\leq m}|t_i-t_{i-1}|\] die \textit{Feinheit} von $Z$. Es gibt dann eine Folge $(Z_l)$ von Zerlegungen, so dass \[ \lim_{l\to\infty} L(c_{Z_l})=L(c) \] und $\lim_{t\to \infty}\delta(Z_t)=0.$ Wir parametrisieren $c_{\vek{x}^l_{i-1}\vek{x}^l_i}$ 
	durch \[[t^l_{i-1},t^l_i]\to \R^n, \qquad t\mapsto \vek{x}^l_{i-1} + \frac{t-t^l_{i-1}}{t^l_{i}-t^l_{i-1}}(\vek{x}^l_i-\vek{x}^l_{i-1})\] mit $\vek{x}^l_i=c(t^l_i)$. Die Länge von $c_{\vek{x}^l_{i-1},\vek{x}^l_i}$ 
	bleibt unverändert, außerdem gilt \[ L(c_{Z_t})=\sum_{i=1}^{m_l}\int_{t^l_{i-1}}^{t^l_i} \norm{\dot{c}_{\vek{x}^l_{i-1},\vek{x}^l_i}(t)} \diff t \] mit \[ \dot{c}_{\vek{x}^l_{i-1},\vek{x}^l_i}(t)=\frac{c(t^l_i)-c(t^l_{i-1})}{t^l_i-t^l_{i-1}}. \] Damit ist \[  \dot{c}_{\vek{x}^l_{i-1},\vek{x}^l_i}(t)-\dot{c}(t)=\frac{1}{t^l_i-t^l_{i-1}}\int_{t^l_{i-1}}^{t^l_i}(\dot{c}(\xi)-\dot{c}(t))\diff \xi \]
	Setzen wir \[ f_l(t):=\dot{c}_{\vek{x}^l_{i-1},\vek{x}^l_i}(t), \] so ist wegen der gleichmäßigen Stetigkeit von $\dot{c}(t)$ in $[a,b]$ \[ \norm{\norm{f_l(t)}-\norm{\dot{c}(t)}}\leq \norm{f_l(t)-\dot{c}(t)}\leq \varepsilon, \] wenn $\delta(Z_l)\leq \delta(\varepsilon)$. Also ist $f_l(t)$ gleichmäßig konvergent und mit dem Konvergenzsatz von Lebesgue ist \[L(c)=\lim_{l\to\infty} L(c_{Z_l})=\lim_{l\to\infty}\int_a^b \norm{f_l(t)}\diff t=\int_a^b\norm{\dot{c}(t)}\diff t.\]
\end{proof}
\begin{de}[Gleichförmig parametrisierte Kurve]
Sei $c:I\to\R^n$ eine parametrisierte $C^k$-Kurve und $t_0\in I$, so die Kurve $c$ \textit{gleichförmig parametrisiert für} $t\geq t_0$, falls ein $C>0$ existiert, so dass \[ \int_{t_0}^t\norm{\dot{c}(\theta)}_2\diff\theta=C(t-t_0). \] D.h. die Länge von $c$ eingeschränkt auf $[t_0,t]$ ist \textit{proportional} zu $t-t_0$.
\end{de}
\begin{lem}
	Wenn $c:I\to\R^n$ eine parametrisierte $C^k$-Kurve, die für $t\geq t_0$ gleichförmig parametrisiert ist, dann gibt es ein $C>0$, so dass \[\norm{\dot{c}(\theta)}_2=C\] für alle $\theta\in[t_0,t]$.
\end{lem}
\begin{bsp}[Gleichförmige Bewegung eines Massepunktes]
	
\end{bsp}
\begin{de}[Bogenlänge]
Sei $c:[a,b]\to\R^n$ eine parametrisierte $C^k$-Kurve. Die Funktion \[ s_c(t):=\int_a^t \norm{\dot{c}(\theta)}\diff\theta, \qquad t\in [a,b]  \] heißt die
\textit{Bogenlänge von $c$}. Wir sagen eine parametrisierte $C^k$-Kurve $c:[a,b]\to \R^n$ ist
\textit{proportional zur Bogenlänge parametrisiert}, falls es ein $C>0$ gibt, so dass \[  \int_a^t \norm{\dot{c}(\theta)}\diff\theta = C(t-a)\] gilt. Eine parametrisierte $C^k$-Kurve $c:[a,b]\to \R^n$ ist \textit{mit Bogenlänge parametrisiert}, falls $C=1$ ist.
\end{de}
\begin{lem}
	Wenn die parametrisierte $C^k$-Kurve $c:[a,b]\to \R^n$ mit Bogenlänge parametrisiert ist, dann ist \[ \norm{\dot{c}(t)}_2=1 \] für alle $t\in[a,b]$.
\end{lem}
\begin{theo}
	Sei $c:[a,b]\to \R^n$ eine parametrisierte $C^k$-Kurve. So ist $c$ genau dann proportional zur Bogenlänge parametrisierbar, falls $\dot{c}(t)\neq \vek{0}$ für alle $t\in[a,b]$ gilt.
\end{theo}
\begin{proof}
	$(\Rightarrow).$ Sei $c:[a,b]\to \R^n$ eine parametrisierte $C^k$-Kurve und nach Bogenlänge parametrisierbar. Differentiation nach $t$ ergibt \[ \frac{\diff}{\diff t} \int_{t_0}^t\norm{\dot{c}(\theta)}_2\diff\theta=C\neq 0.\]
	$(\Leftarrow).$ Sei $\dot{c}\neq \vek{0}$ auf ganz $[a,b]$. Dann definiert \[s_c(t):=\int_a^t \norm{\dot{c}(\theta)}\diff\theta, \qquad t\in [a,b] \] eine zulässige Parametertransformation. Somit ist für $\tilde{c}=c\circ s_c^{-1}$ \[\norm{\frac{\diff}{\diff \theta}\tilde{c}(\theta)}=\norm{\dot{c}(s_c^{-1}(\theta))}\frac{1}{\norm{\dot{c}(s_c^{-1}(\theta))}}=1\]
\end{proof}
\begin{de}[Reguläre Kurve]
	Es sei $c:I\to\R^n$ eine parametrisierte $C^k$-Kurve. Wir nennen die Kurve $c$ \textit{regulär}, falls für alle $t\in I$ \[\dot{c}(t)\neq \vek{0}\in \R^n\] gilt.
\end{de}
\begin{bsp}
	
\end{bsp}
\begin{de}[Tangenete]
	Für eine reguläre parametrisierte $C^k$-Kurve $c:I\to\R^n$ heißt die durch
	\begin{align*}
	g:\R&\to\R^n\\
	\lambda&\mapsto c(t)+\lambda\dot{c}(t)
	\end{align*}
	definierte Gerade im $\R^n$ \textit{Tangente an $c$ im Punkt $c(t)$}. Der Vektor $\dot{c}(t)$ heißt \textit{Tangentialvektor an $c$ im Punkt $c(t)$}.
\end{de}
\section{Vektorfelder und Integralkurven}
\begin{de}[Gebiet]
	Sei $X$ ein topologischer Raum. Ein \textit{Gebiet} $G\subseteq X$ ist eine offene, nichtleere und zusammenhängende Teilmenge von $X$. Ein Gebiet $G\subseteq \R^n$ heißt \textit{sternförmig}, falls es ein $\vek{x}_0\in G$ gibt, so dass für alle $\vek{x}\in G$ die Strecke \[[\vek{x}_0\vek{x}]=\lbrace \vek{x}_0+t(\vek{x}-\vek{x}_0)\mid t\in [0,1]\rbrace\] eine Teilmenge von $G$ ist. Das Gebiet $G$ nennen wir \textit{konvex}, falls für alle $\vek{x},\vek{y}\in G$ und alle $t\in \R$ mit $0\leq t\leq 1$ gilt, dass \[  t\vek{x}+(1-t)\vek{y}\in G \] ist.
\end{de}
\begin{de}[Gewöhnliche Differentialgleichung]
Sei $G\subseteq \R^n$ ein Gebiet und $F\in C^0((a,b)\times G,\R^n)$. Eine Funktion $u\in C^1((\alpha,\beta),G)$ mit $a\leq \alpha <\beta\leq b$ heißt \textit{Lösung der durch $F$ definierten gewöhnlichen Differentialgleichung erster Ordnung mit Anfangswert $u_0\in G$ in $t_0\in (\alpha,\beta)$}, wenn gilt: 
\begin{align*}
	\dot{u}(t)&=F(t,u(t)), \quad t\in (\alpha,\beta),\\
	u(t_0)&=u_0.
\end{align*} 
\end{de}
\begin{theo}[Satz von Picard-Lindelöf]
	Sei $-\infty\leq a<b\leq \infty$, $G\subseteq \R^n$ ein Gebiet und $F\in C^0((a,b)\times G,\R^n)$. Es gilt:
	\begin{itemize}
		\item[(i)] Zu jedem $t_0\in (a,b)$ und jedem $\vek{f}_0\in G$ gibt es ein $\varepsilon>0$ mit $(t_0-\varepsilon,t_0+\varepsilon)\subseteq (a,b)$ und eine Umgebung $U$ von $\vek{f}_0\in G$, so dass für $u_1\in U$ eine Lösung \[u\in C^1((t_0-\varepsilon,t_0+\varepsilon),G)\] des Problems \begin{align*}
		\dot{u}(t)&=F(t,u(t)),\quad t\in (t_0-\varepsilon,t_0+\varepsilon),\\u(t_0)&=\vek{u}_1
		\end{align*}
		existiert.
		\item[(ii)] Erfüllt $F$ in jedem Punkt $(t_0,u_0)$ die Bedingung, dass zu jedem $(t_0,\vek{u}_0)$ eine Umgebung $(t_0-\varepsilon,t_0+\varepsilon)\times U\subset(a,b)\times G$ derart existiert, dass für $t_0-\varepsilon<t<t_0+\varepsilon$ und $\vek{x}_1,\vek{x}_2\in U$ \[  \norm{F(t,\vek{x}_1)-F(t,\vek{x}_2)}\leq M\norm{\vek{x}_1-\vek{x}_2} \] gilt mit einer nur von $\varepsilon$ und $U$ abhängigen Konstanten $ M$, so sind die nach $(i)$ existierenden Lösungen für jeden Anfangswert eindeutig bestimmt.
		\item[(iii)] Gilt sogar $F\in C^k$ mit $k\geq 1$, so gilt für die Lösung $u$ des Anfangswertproblems $u\in C^{k+1}$.
	\end{itemize}
\end{theo}
\begin{de}[Dynamisches System]
	Da durch diese Differentialgleichungen oft die zeitliche Entwicklung bzw. die Dynamik vieler natürlicher Phänomene beschrieben werden nennen wir sie auch \textit{dynamische Systeme}. Hängt $F$ nicht explizit von der Zeit ab, d.h. \[ F(t,\vek{x})=\tilde{F}(\vek{x}), \] so ist \[ \dot{u}(t)=\tilde{F}(u(t)),  \] so nennen wir das System \textit{autonom}. Alle Systeme für die das nicht gilt heißen \textit{nicht autonom}.
\end{de}
\begin{de}[$C^k$-Vektorfeld]
	Sei $G\subseteq \R^n$ ein Gebiet. Eine Funktion $F\in C^k(G,\R^n)$, $k\in \mathbb{Z}_+$, heißt ein $k$-fach differenzierbares Vektorfeld (oder kürzer $C^k$-Vektorfeld) in $G$.
\end{de}
\begin{de}[Integralkurve]
Sei $F$ ein $C^k$-Vektorfeld in $G$, $k\geq 1$, und $x_0\in G$. Jede $C^1$-Lösung $c:[a,b]\to G$ der durch $F$ definierten Differentialgleichung mit $-\infty\leq a<0<b\leq \infty$ und $c(0)=\vek{x}_0$ heißt eine Integralkurve von $F$ durch $\vek{x}_0$.
\end{de}
\begin{bsp}
	Wir betrachen folgende Differentialgleichung:
	\[  \dot{u}(t)= \frac{2u(t)}{t}.\] Also ist $f(t,u(t))=\frac{2u(t)}{t}$ für $x=u(t)$, somit ergibt sich $f(t,x)=\frac{2x}{t}$. Diese Differentialgleichung ist also nicht autonom. Wir sehen, dass, wenn der Graph einer Lösung durch den Punkt $(t,x)$ läuft, dieser dort die Steigung $\frac{2x}{t}$ hat. Somit lässt sich jedem Punkt $(t,x)$ in einem Gebiet $G\subseteq\R^2$ ein Vektor mit Steigung $\frac{2x}{t}$ zuordnen. Somit haben wir durch \[F:G\to \R^2,\qquad (t,x)\mapsto (1,\frac{2x}{t})\] ein Vektorfeld definiert. Damit sind alle möglichen Lösungen $u$ der Differentialgleichung durch $F$ definiert. Die allgemeine Lösung der Differential ist $u(t)=kt^2$. Für einen konkreten Anfangswert $u(1)=5$ ist somit $u(t)=5t^2$ die Lösung der Differentialgleichung. Damit ist \[ c:[1,\infty)\to \R^2,\qquad t\mapsto (t,5t^2) \] die gesuchte Integralkurve zu $F$ durch $(1,5)$, denn \[ \dot{c}(t)=(1,10t)=(1,\frac{2\cdot 5t^2}{t})=(1,\frac{2u(t)}{t})=(1,f(t,u(t)))=F(t,u(t)) \]
\end{bsp}
\begin{de}[Stationärer Punkt]
	Sei $F$ ein $C^k$-Vektorfeld in $G$. Die Punkte $x\in G$ mit $F(x)=0$ heißen \textit{stationären Punkte von $F$}.
\end{de}
\begin{theo}[Fundamentalsatz über die Integralkurve]
	Sei $G\subseteq \R^n$ ein Gebiet und $F$ ein $C^k$-Vektorfeld in $G$, $k\geq 1$. Dann gibt es zu jedem $\vek{x}\in G$ eine ausgezeichnete Integralkurve $c_\vek{x}:(a_\vek{x},b_\vek{x})\to G$ durch $\vek{x}$ mit folgenden Eigenschaften: \begin{itemize}
		\item[(i)] $-\infty\leq a_k<0<b_k\leq\infty$,
		\item[(ii)] $c_\vek{x}\in C^{k+1}((a_\vek{x},b_\vek{x}),G)$,
		\item[(iii)] Ist $c:(a,b)\to G$ eine Integralkurve durch $\vek{x}$, so ist $a_\vek{x}\leq a <b \leq b_\vek{x}$ und $c=c_\vek{x}\mid_{(a,b)}$,
		\item[(iv)] Zu jedem $\vek{x}\in G$ und $t\in (a_\vek{x},b_\vek{x})$ gibt es eine Umgebung $U$ von $\vek{x}$ in $G$ und ein $\varepsilon>0$, so dass die Abbildung \begin{align*}
		 U\times (t-\varepsilon,t+\varepsilon) &\longrightarrow G\\
		 (\vek{y},s)\mapsto c_\vek{y}(s)
		\end{align*}
		 von der Klasse $C^k$.
	\end{itemize}
\end{theo}
\begin{proof}
	Sei $C_\vek{x}$ die Menge aller Integralkurven von $F$ durch $\vek{x}$. Für zwei Kurven $c_1,c_2$ ist durch $c_1\prec c_2 :\Leftrightarrow (a_1,b_1)\subset (a_2,b_2)$ eine Halbordnung auf $C_\vek{x}$ definiert. Nach dem Satz von Picard-Lindelöf ist $c_1=c_2$ auf $(a_1,b_1)$. Es bleibt nur zu zeigen, dass $C_\vek{x}$ bezüglich der Halbordnung ein maximales Element hat. Sei $D\subset C_\vek{x}$ und $c\in D$ mit Definitionsintervall $(a_c,b_c)$. Somit ist \[(a_\vek{x},b_\vek{x}):=\bigcup_{c\in D}(a_c,b_c).\] Setzen wir \[ c_\vek{x}(t):=c(t),\qquad t\in (a_c,b_c). \] Aus dem Satz von Picard-Lindelöf folgt, dass $c_\vek{x}$ wohldefiniert und eine Integralkurve durch $\vek{x}$ ist, womit nach dem Lemma von Zorn $C_\vek{x}$ ein maximales Element besitzt.
\end{proof}
\begin{de}[Vollständiges $C^k$-Vektorfeld]
	Sei $G\subseteq \R^n$ ein Gebiet und $F$ ein $C^k$-Vektorfeld in $G$, $k\geq 1$. $F$ heißt \textit{vollständig}, wenn $(a_\vek{x},b_\vek{x})=\R$ für jedes $\vek{x}\in G$.
\end{de}
\begin{de}[Fluss]
	Sei $F$ ein vollständigges $C^1$-Vektorfeld auf einem Gebiet $G\subseteq \R^n$. Nach dem Satz von Picard-Lindelöf gibt es für jedes $\vek{x}_0 \in G$ eine eindeutige maximale Lösung $c_{\vek{x}_0}:(a_{\vek{x}_0},b_{\vek{x}_0})\to \R$ der Differentialgleichung \[ \dot{\vek{x}}(t)=F(\vek{x}),\quad \vek{x}(0)=\vek{x}_0.\] Die Abbildung $\Phi(t,\vek{x}):=c_\vek{x}(t)$ heißt \textit{Fluss des Vektorfeldes $F$}.
\end{de}
\begin{lem}[Eigenschaften des Flusses]
	Sei $\Phi:\R\times G\to G$ der Fluss eines vollständigen $C^k$-Vektorfeldes $F$ in $G\subseteq \R^n$. Dann erfüllt der Fluss $\Phi_t$ folgende Eigenschaften:
	\begin{itemize}
		\item[(i)] Für jedes $t\in \R$ ist $\Phi_t:G\to G$ eine stetige Funktion.
		\item[()] Für den Zeitpunkt  $t=0$ ist $\Phi_0$ die Identität $\operatorname{id}_G$ auf $G$.
		\item[(iii)] Für $s,t\in \R$ ist $\Phi_t\circ\Phi_s=\Phi_{t+s}$.
	\end{itemize}
\end{lem}
\begin{de}[Modul über dem Ring $C^k(G)$]
	Sei $G\subseteq \R^n$ ein Gebiet und es seien  $F,H$ zwei $C^k$-Vektorfelder in $G$ und $f\in C^k(G,\R)$ ein Skalarfeld. So definieren wir durch skalare Multiplikation \[ f(F(\vek{x})):=f(\vek{x})F(\vek{x})\] ein neues $C^k$-Vektorfeld $fF$ und durch Vektoraddition \[(F+H)(\vek{x}):=F(\vek{x})+H(\vek{x})\] ebenfalls ein neues $C^k$-Vektorfeld $F+H$ auf $G$. Der Vektorraum aller $C^k$-Vektorfelder auf $G$ ist ein \textit{Modul über dem Ring} $C^k(G)$.
\end{de}
\begin{de}[Lineares stetiges Vektorfeld]
Ein $C^0$-Vektorfeld $F$ im $\R^n$ heißt \textit{linear}, falls es eine lineare Funktion in $\vek{x}\in \R^n$ ist, d.h. \[ F(\la \vek{x}+\mu \vek{y})=\la F(\vek{x})+\mu F(\vek{y}). \] 
\end{de}
Jedes lineare Vektorfeld $F$ in $\R^n$ gehört also eine Matrix $(F_{ij})$, so dass \[ F_i(x_1,...,x_n)=\sum_{j=1}^{n}F_{ij}x_j,\qquad 1\leq i\leq n. \] Für $n=1$ und $F:\R\to \R$ linear, so ist \[F(x)=ax\] für $a\in \R$. Somit ergibt sich für die Differentialgleichung \[ \dot{c}(t)=ac(t), c(0)=x_0 \] die Lösung \[c(t)=x_0e^{at}=e^{at}x_0. \] Für den mehrdimensionalen Fall \[\dot{c}(t)=F(c(t)),\qquad c(0)=\vek{x}_0 \] erhalten wir so $c(t)=e^{tF}(\vek{x}_0)$. Es stellt sich nun die Frage wie sich $e^{tF}$ praktisch berechnen lässt.
Wir betrachten zwei Spezialfälle:\\
$(i)$ Sei $F$ eine $n\times n$-Matrix. $F$ ist \textit{ähnlich zu einer Diagonalmatrix} $D=(\delta_{ij}\lambda_j)$, d.h. es gibt eine invertierbare Matrix $T$ mit \[F=TDT^{-1}.\] Somit sind $\lambda_1,..,\lambda_n$ gerade die Eigenwerte von $F$ und es gibt eine Basis $e_1,...,e_n$ von $\R^n$ mit $e_i$ ist Eigenvektor von $F$ zum Eigenwert $\lambda_i$. Somit ist \[e^{tF}=Te^{tD}T^{-1},\] während \[ e^{tD}=(\delta_{ij}e^{t\lambda_j}) \]
$(ii)$ F ist \textit{nilpotent}, d.h. . 
Zu dieser nilpotenten $n \times n$-Matrix existiert ein $k \leq n$ mit $F^k = 0$, also $F^n = 0$. 
Dann wird $e^{tF}$ eine Matrix, deren sämtliche Koeffizienten Polynome in $t$ sind vom Grad kleiner gleich $n-1$. Eine nilpotente Matrix ist immer ähnlich zu einer oberen Dreiecksmatrix, die auf der Diagonalen nur Nullen hat.\\
$(iii)$ Sei $A$ Element eines komplexer Vektorraums, so gibt es eine Zerlegung \[A=S+N,\qquad SN=NS\] mit $S$ einer diagonalisierbaren und $N$ einer nilpotenten Matrix. Da $S$ und $N$ kommutieren, gilt \[ e^A=e^{S+N}=e^Se^N.\] Für einen reellen Vektorraum gilt folgender Satz:
\begin{theo}[Normalformensatz]
	Sei $F=(F_{ij})$ ein lineares Vektorfeld auf $\R^n$. Dann ist $F$ vollständig und der Fluss $(\Phi_t)$ von $F$ ist gegeben durch \[\Phi(t,\vek{x})=e^{tF}(\vek{x}),\qquad t\in \R,\vek{x}\in \R^n.\] Dabei ist \[e^{tF}:=\sum_{j=0}^\infty\frac{t^j}{j!}F^j=\left(  \sum_{k=1}^n a_{ij}^k(t) e^{\lambda_k t} \right),\]
	wobei $\lambda_1,\lambda_n$ die Eigenwerte von $F$ sind und $a_{ij}^k(t)$ Polynome in $t$ vom Grad kleiner gleich $n-1$.
\end{theo}
\begin{de}[Gradientenfeld]
	Sei $G\subseteq \R^n$ ein Gebiet und $F$ ein $C^k$-Vektorfeld in $G$. $F$ ist ein \textit{Gradientenfeld}, falls es eine $C^{k+1}$-Funktion $\varphi:G\to \R$ gibt mit \[F(\vek{x})=\nabla \varphi(\vek{x}).\] Wir sagen zu $\varphi$ auch \textit{Skalarpotential}.
\end{de}
\begin{lem}[Integrabilitätsbedingungen]
	Sei $G\subseteq \R^n$ ein Gebiet und $F$ ein $C^k$-Vektorfeld in $G$, $k\geq 1$.  Ist $F$ ein Gradientenfeld, so sind die  Integritätsbedingungen
	 \[  \partial_j F_i(\vek{x})=\partial_i F_j(x),\qquad \vek{x}\in G, 1\leq i,j\leq n \] 
	 erfüllt. Ist das Gebiet zusammenhängend, so gilt sogar, dass $F$ ein Gradientenfeld ist genau dann, wenn alle Integrabilitätsbedingung erfüllt sind.
\end{lem}
\begin{proof}
	Sei $F$ ein Gradientenfeld, so gibt es also $\varphi\in C^{k+1}(G)$ mit \[  F_i(\vek{x})=\partial_i \varphi(\vek{x}). \] Da partielle Ableitungen vertauschbar sind gilt \[ \partial_j F_i(\vek{x})=\partial_j\partial_i\varphi(\vek{x})=\partial_i\partial_j\varphi(\vek{x})=\partial_i F_j(\vek{x}), \]
	$\vek{x} \in G$, $1\leq i,j\leq n.$
\end{proof}
\begin{de}[Hamilton-Vektorfeld]
	Ein $C^k$-Vektorfeld $F$ im $\R^{2n}$ heißt \textit{Hamilton-Vektorfeld}, falls es eine $C^{k+1}$-Funktion $H$ im $\R^{2n}$ gibt mit \[ F(\vek{x})=I(\nabla_\vek{x}H(\vek{x})),\qquad I:=\begin{pmatrix}
	0 & I_n\\
	-I_n & 0
	\end{pmatrix}. \] $H$ heißt dann die zu $F$ gehörende \textit{Hamilton-Funktion}. 
\end{de}
\begin{bsp}
	
\end{bsp}
\section{Kurvenintegrale}
Wir erinnern zunächst daran, dass eine endliche Kurve im $\R^n$  Hausdorff-Dimension $1$ hat. D.h. die Länge einer endlichen Kurve  entspricht dem Hausdorff-Maß von $\mathcal{H}^1$ der Dimension 1.
\begin{de}[Kurvenintegral]
	Sei $f:\R^n\to \R$ ein $C^0$-Skalarfeld und $c:[a,b]\to \R^n$ eine parametrisierte $C^1$-Kurve ( oder auch stückweise $C^1$). Dann ist das \textit{Kurvenintegral erster Art} entlang der Kurve $c$ definiert als \[  \int_c f \diff s:=\int_a^b f(c(t))\norm{\dot{c}(t)}_2\diff t. \] Hierbei  ist $s$ die Länge eines Kurvenelements. Sei $F:\R^n\to \R^n$ ein $C^0$-Vektorfeld und $c:[a,b]\to \R^n$ eine parametrisierte $C^1$-Kurve. Dann ist das \textit{Kurvenintegral zweiter Art} entlang der Kurve $c$ definiert als \[ \int_{c} F \diff s:= \int_a^b \ip{F(c(t))}{\dot{c}(t)}\diff t. \]
\end{de}
\begin{lem}
	Sei $G\subseteq \R^n$ ein Gebiet, $F$ ein $C^0$-Vektorfeld in $G$ und $c_1:[a,b]\to G$ eine reguläre Kurve. Ist $\varphi:[\alpha,\beta]\to[a,b]$ eine Parametertransformation von $c_1$, dh. $c_2=c_1\circ\varphi$, so gilt \[  \int_{c_2} F \diff s = \pm \int_{c_1}F\diff s . \]
\end{lem}
\begin{bsp}
	
\end{bsp}
\begin{de}[Energie einer regulären Kurve]
	Sei $c:[a,b]\to \R^n$ eine reguläre Kurve, so nennen wir
	\[ E(c):=\int_a^b\norm{\dot{c}(t)}^2\diff t \] die \textit{Energie der Kurve}.
\end{de}
\begin{lem}
	Das Kurvenintegral ist linear, d.h. für zwei $C^0$-Vektorfelder $V,W$ ist \[ \int_c (\alpha V + W)\diff s=\alpha \int_c V \diff s+\int_c W\diff s. \]  Es ist bis auf das Vorzeichen invariant bzgl. der Durchlaufrichtung der Kurve,  d.h. ob die Kurve positiv oder negativ durchlaufen wird. Insbesondere ist für eine $C^0$-Vektorfeld $F$ die aus den regulären Kurven $c_1$ und $c_2$ zusammen gesetzte Kurve $c=c_1\star c_2$ das Kurvenintegral durch \[  \int_{c_1\star c_2} F \diff s=\int_{c_1}F\diff s+\int_{c_2}F \diff s \] gegeben.
\end{lem}
\begin{proof}
	Die erste zwei Aussagen folgen direkt aus den Eigenschaften des abstrakten Integrals. Sei $c_1:[a,b]\to \R^n$ und $c_2:[c,d]\to \R^n$ zwei parametrisierte reguläre Kurven. Dann ist  \begin{align*}
	\int_{c_1\star c_2}F \diff s &= \int_a^{b+(d-c)} \ip{F(c_1\star c_2(t))}{\dot{(c_1\star c_2)}(t)}\diff t\\
	&=\int_a^b\ip{F(c_1(t))}{\dot{c}_1(t)}\diff t + \int_a^{b+(d-c)}\ip{F(t-b+c)}{\dot{c}_2(t-b+c)}\diff t\\
	&=\int_{c_1}F\diff s+\int_{c_2}F \diff s
	\end{align*}
\end{proof}
\begin{de}[Wegunabhängig integrierbar]
	Sei $G\subseteq \R^n$ ein Gebiet und $F$ ein $C^0$-Vektorfeld auf $G$. Dann heißt $F$ \textit{wegunabhängig integrierbar}, falls für jede parametrisierte geschlossene stückweise $C^1$-Kurve $c$ in $G$ gilt, dass \[ \int_c F \diff s=0. \]
\end{de}
\begin{theo}
	Sei $G\subseteq \R^n$ ein Gebiet und $F$ ein $C^0$-Vektorfeld auf $G$. $F$ ist genau dann in $G$ wegunabhängig integrierbar, wenn $F$ ein Gradientenfeld ist, d.h. wenn es $\varphi\in C^1(G)$ gibt mit \[ F=\nabla\varphi.\] Insbesondere ist ein $C^1$-Vektorfeld $F$ auf einen sternförmigen Gebiet genau dann ein Gradientenfeld, wenn die Integrabilitätsbedingungen \[ \partial_{j}F_i=\partial_iF_j \qquad \textnormal{ für } i,j\in\left\lbrace1,...,n\right\rbrace \] erfüllt sind.
\end{theo}

\section{Satz von Gauß und die Formeln von Green}
Der Fundamentalsatz der Analysis sagt uns, dass wir unter gewissen Voraussetzung, wie der stetigen Differenzierbarbeit an die Stammfunktion des Integranden und der Integration auf einem kompakten Intervall, das eindimensionale Integral durch die Auswertung an den zwei Randpunkten der Stammfunktion berechnen können, also durch Integrale der Dimension 0. Die Frage ist nun, ob es einen vergleichbaren Zusammenhang zwischen Integralen der Dimension zwei und Integralen der Dimension 1 gibt, oder allgemeiner Integrale der Dimension $n$ und Integralen der Dimension $n-1$. Wir untersuchen diesen Zusammenhang zunächst für spezielle Gebiete $G\subseteq \R^2$.  
\begin{de}[Gebiete erster und zweiter Art]
	Seien $f_1,f_2:[a,b]\to \R$ zwei $C^1$-Funktion mit $f_1(x)<f_2(x)$ für $x\in[a,b]$. Dann nennen wir \[G:=\left\lbrace (x,y)\in\R^2\mid a<x<b, f_1(x)<y<f_2(x)\right\rbrace\] ein \textit{Gebiet 1. Art}. 
	Seien $g_1,g_2:[a,b]\to \R$ zwei $C^1$-Funktionen mit $g_1(y)<g_2(y)$ für $y\in[a,b]$. Dann nennen wir \[G:=\left\lbrace (x,y)\in\R^2\mid a<y<b, g_1(x)<x<g_2(x)\right\rbrace\] ein \textit{Gebiet 2. Art}. 
\end{de}
\begin{de}[Randintegral im $\R^2$]
	Sei $G\subseteq \R^2$ ein Gebiet, dessen Rand $\partial G$ das Bild einer regulären geschlossenen stückweise stetig differenzierbare Jordankurve $c$ ist. Wird $c$ so orientiert, dass $J(\dot{c})$ immer ins Innere von $G$ zeigt, so ist für ein in einer Umgebung von $\partial G$ definiertes $C^0$-Vektorfeld F das \textit{Randintegral} definiert durch \[ \int_{\partial G} F \diff \mathcal{H}^1=\int_c F \diff s \]
\end{de}
Sei $h:\overline{G}\to \R$ stetig und $G$ ein Gebiet 1. Art, dann ist 
\[ \int_G h(x,y)\diff(x,y)=\int_a^b \left(  \int_{f_1(x)}^{f_2(x)} h(x,y)\diff y \right)\diff x \]
das bekannte Gebietsintegral. Das Integral über den Rand ist ein Kurvenintegral über die Randkurve $c$ mit bspw. folgender Parametrisierung $c:=c_1\circ c_2\circ c_3^{-1}\circ c_4^{-1}$ mit
\begin{align*}
&c_1:[a,b]\to \R^2,\qquad t\mapsto(t,f_1(t))\\
&c_2:[0,1]\to \R^2,\qquad t\mapsto (b,f_1(b)+t(f_2(b)-f_1(b)))\\
&c_3:[a,b]\to\R^2,\qquad t\mapsto (t,f_2(t))\\
&c_4:[0,1]\to\R^2,\qquad t\mapsto (a,f_1(a)+t(f_2(a)-f_1(a)))
\end{align*}
Das Kurvenintegral ist für Vektorfelder definiert, das Gebietsintegral für Skalarfelder. Sei also $F$ ein $C^1$-Vektorfeld auf $\R^2$. Dann suchen wir eine Funktion $f$, so dass \[ \int_c F \diff s=\int_G f(x,y)\diff(x,y). \]
Erfüllt $F$ die Integrabilitätsbedingungen, so ist \[\partial_2 F_1(x,y)=\partial_1F_2 \quad \textnormal{oder} \quad \left(  \partial_2F_1-\partial_1F_2 \right)(x,y)=0,\] damit ist $\int_c F\diff s=0$, da $c$ eine geschlossene Kurve ist. Also gilt in diesem Fall \[ \int_c F\diff s=\int_G  \left(  \partial_2F_1-\partial_1F_2 \right)(x,y) \diff(x,y).\] Die Frage ist nun, ob diese Identität immer gilt. Dazu formen wir um: \[ \int_G  \left(  \partial_2F_1-\partial_1F_2 \right)(x,y) \diff(x,y)=\int_a^b\int_{f_1(x)}^{f_2(x)}\left( \partial_2F_1-\partial_1F_2 \right)(x,y)\diff y\diff x. \]
Mit dem Fundamentalsatz der Analysis ist \[  \int_a^b \int_{f_1(x)}^{f_2(x)}\partial_2 F_1(x,y)\diff y\diff x=\int_a^b F_1(x,f_2(x)))-F_1(x,f_1(x))\diff x.\] Weiter ist $\frac{\diff}{\diff x}\int_{f_1(x)}^{f_2(x)}F_2(x,y)\diff y=F_2(x,f_2(x))f'_2(x)-F_2(x,f_1(x))f'_1(x)+\int_{f_1(x)}^{f_2(x)}\partial_1 F_2(x,y)\diff y.$ Damit ist 
\begin{align*}
\int_G \left( \partial_2F_1-\partial_1F_2 \right)(x,y)\diff(x,y)&=\int_a^b(F_1(x,f_2(x))-F_1(x,f_1(x)))\diff x\\
&+\int_a^b(F_2(x,f_2(x))f'_2(x)-F_2(x,f_1(x))f'_1(x))\diff x\\
&-\int_{f_1(b)}^{f_2(b)}F_2(b,y)\diff y+\int_{f_1(a)}^{f_2(a)}F_2(a,y)\diff y\\
&=\int_a^b[(F_1(x,f_2(x))+F_2(x,f_2(x))f'_2(x))\\&-(F_1(x,f_1(x))+F_2(x,f_1(x))f'_1(x))]\diff x\\
&+\int_{f_1(a)}^{f_2(a)}F_2(a,y)\diff y-\int_{f_1(b)}^{f_2(b)}F_2(b,y)\diff y.
\end{align*}
Damit ergibt sich \[\int_G \left(  \partial_2 F_1-\partial_1 F_2 \right)(x,y)\diff(x,y)=\int_{c_3}F\diff s-\int_{c_1}F\diff s+\int_{c_4}F\diff s-\int_{c_2}F\diff s \] oder \[ \int_G\left(  \partial_1F_2-\partial_2F1 \right)(x,y)\diff(x,y)=\int_c F \diff s. \]
Fassen wir unsere Ergebnis in einem Lemma zusammen. 
\begin{lem}
	Sei $G\subseteq \R^2$ ein Gebiet erster Art und $F$ ein  $C^1$-Vektorfeld in einem Gebiet $G'\subseteq \R^2$ mit $\overline{G}\subseteq G'$. Ist $c$ die Raumkurve von $G$ mit der oben angegebenen Parametrisierung, so gilt \[  \int_G \left(  \partial_1 F_2-\partial_2F_1 \right)(x,y)\diff (x,y)=\int_c F \diff s. \]
\end{lem}
Wir versuchen nun die Aussage auf beliebige Gebiete zu erweitern.
In einem beliebigen Gebiet $G\subseteq \R^2$, dessen Rand Bild einer regulären geschlossenen stückweise stetig differenzierbare Jordankurve $c$ ist. Wählen wir eine doppelpunktfreie reguläre stückweise $C^1$-Kurve $c:[a,b]\to \R^2$ mit $c(a),c(b)\in\partial G$. Dann zerfällt $G$ in zwei Gebiete $G_1$ und $G_2$, die von $\partial G$ und $c$ berandet werden. Gilt vorheriges Lemma nun für $G_1$ und $G_2$, so auch für ganz $G$. Sei der Rand $\partial G_1$ parametrisiert durch $c_1\circ c^{-1}$ und der Rand $\partial G_2$ durch $	c\circ c_2$.
Also gilt für ein geeignetes Vektorfeld $F$ 
\begin{align*}
\int_G(\partial_1F_2-\partial_2F_1)(x,y)\diff(x,y)&=\int_{G_1}\partial_1F_2-\partial_2F_1)(x,y)\diff(x,y)+\int_{G_2}\partial_1F_2-\partial_2F_1)(x,y)\diff(x,y)\\
&=\int_{\partial G_1}F \diff \mathcal{H}^1+\int_{\partial G_2}F \diff \mathcal{H}^1=\int_{c_1\circ c^{-1}}F\diff s+\int_{c\circ c_2}F\diff s\\
&=\int_{c_1}F\diff s + \int_{c_2} F \diff s= \int_{c_1\circ c_2}F \diff s=\int_{\partial G} F \diff s.
\end{align*}
Lässt ein Gebiet $G$ in endlich viele Gebiete erster bzw zweiter Art aufteilen, so nennen wir es \textit{zulässig}. Damit erhalten wir folgende bekannten Satz für Gebiete im $\R^2$.
\begin{theo}[Satz von Gauß-Green]
	Sei $G\subseteq \R^2$ ein zulässiges Gebiet und $F$ ein $C^1$-Vektorfeld in einem Gebiet $G'\subseteq \R^2$ mit $G\subseteq G'$. Dann gilt \[ \int_G \left(  \partial_1 F_2-\partial_2 F_1 \right)(x,y)\diff (x,y)=\int_{\partial G} F \diff s. \]
\end{theo}
Der Satz ist allerdings nicht so einfach ohne Weiters auf höhere Dimensionen verallgemeinerbar. Dazu sind einige Modifikationen notwendig.
\begin{de}[Innere und äußere Normale]
		Sei $G\subseteq \R^2$ ein zulässiges Gebiet und $F$ ein $C^1$-Vektorfeld. Dann ist \[ (x,y)\mapsto J(F(x,y))=(-F_2(x,y),F_1(x,y)) \] ein $C^1$-Vektorfeld. Der Satz von Gauß-Green liefert \[ \int_G (\partial_1 F_1+\partial_2F_2)(x,y)\diff (x,y)=\int_{\partial G}J(F)\diff s. \] Sei $c:[a,b]\to \R^2$ eine stückweise reguläre Kurve, die den Rand von $G$ parametrisiert. Damit ist \[\int_{\partial G} J(F)\diff \mathcal{H}^1=\int_a^b \ip{J(F)}{(c(t),\dot{c}(t))}\diff t=\int_a^b \ip{F(c(t))}{-J\left(  \frac{\dot{c}(t)}{\norm{\dot{c}(t)}} \right)}\norm{\dot{c}(t)}\diff t.\] Den Einheitsvektor \[ n_i(c(t)):=J\left(  \frac{\dot{c}(t)}{\norm{\dot{c}(t)}} \right) \] nennen wir die \textit{innere Normale von $\partial G $ im Punkt $c(t)$}. Analog nennen wir \[ n_a(c(t)):=-J\left(  \frac{\dot{c}(t)}{\norm{\dot{c}(t)}} \right) \] \textit{äußere Normale von $\partial G$ im Punkt $c(t)$.} 
\end{de}
Damit ist \[  \int_{\partial G} J(F) \diff \mathcal{H}^1=\int_a^b \ip{F(c(t))}{n_a(c(t))}\norm{\dot{c}(t)}\diff t \]   
Setzen wir $f(x):=\ip{F(x)}{n_a(x)}$ für $x\in\partial G$, so folgt mit der Definition des Kurvenintegrals
\[ \int_{\partial G} J(F) \diff \mathcal{H}= \int_a^b \ip{F(c(t))}{n_a(c(t))}\norm{\dot{c}(t)}\diff t =\int_a^b f(c(t))\norm{\dot{c}(t)}\diff t=\int_c f \diff s.\] Erinnern wir uns nun daran, dass \[  \int_G (\partial_1 F_1+\partial_2F_2)(x,y)\diff (x,y)=\int_{\partial G}J(F)\diff s \] ist und allgemein die Divergenz eines Vektorfeldes definiert ist durch \[ \operatorname{div} F(x):=\sum_{j=1}^n \partial_jF_j(x).\] Damit ergibt sich folgender wichtiger Satz.
\begin{theo}[Satz von Gauß]
	ei $G\subseteq \R^2$ ein zulässiges Gebiet, d.h. $\partial G$ ist Bild einer regulären geschlossenen stückweisen $C^1$-Jordankurve und $F$ ein $C^1$-Vektorfeld in einem Gebiet $G'\subseteq \R^2$ mit $G\subseteq G'$. Dann ist \[  \int_G \operatorname{div} F \diff \mathcal{L}^2(x,y)=\int_{\partial G} \ip{F}{n_a} \diff s. \]
\end{theo}
Dieser Satz lässt sich einfacherer verallgemeinern. Das Gebietsintegral entspricht dem $n$-dimensionalen Lebesgue oder  Riemann-Integral. Die rechte Seite entspricht einem Oberflächenintegral der Dimension $n-1$. Die einzige Schwierigkeit bestünde nun nur noch darin für den $n$-dimensionalen Fall, die äußere Normale bzw. das entsprechende Maß für die Oberflächenelemente des abschnittsweise glatten Randes zu bestimmen, im allgemeinen ist dies das $(n-1)$-dimensionale Hausdorffmaß, welches für viele Anwendungsfälle den bekannten Formeln zur Oberflächenberechnung entspricht. 
\begin{theo}[Formeln von Green] Sei $G\subseteq \R^n$ ein zulässiges Gebiet und seien $\varphi,\psi$ zwei $C^2$-Funktionen in einer Umgebung von $G$. Dann gelten folgende Formeln: 
	\begin{align*}
	\int_G \varphi \Delta \psi \diff \mathcal{L}^n(\vek{x})&=\int_{\partial G}\varphi \partial_{n_a}\psi \diff \mathcal{H}^{n-1}-\int_G\ip{\nabla \varphi}{\nabla \psi}\diff \mathcal{L}^n(\vek{x})\\
	\int_G (\varphi \Delta \psi-\psi\Delta\varphi) \diff \mathcal{L}^n(\vek{x})&=\int_{\partial G}\left(  \varphi\partial_{n_a}\psi-\psi\partial_{n_a}\varphi \right)\diff \mathcal{H}^{n-1}
	\end{align*}
	
\end{theo}






\end{document}     